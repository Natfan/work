\documentclass[a4paper,12pt]{article}

\usepackage{graphicx}

%\usepackage[latin1]{inputenc}
%\usepackacge{inputenc}
\usepackage{tikz}

\begin{document}
\title{Unit XXVI Assignment II}
\author{Nathan Windisch}
\date{March 2017}
\maketitle
\pagenumbering{roman}
\tableofcontents
\newpage
\pagenumbering{arabic}

\section{Sequences and Series}
The following are answers to questions set for the first task.
\subsection{nth and 17th Terms}
Find a formula for the \textsc{nth} term of this sequence and find the \textsc{17th} term using your \textsc{nth} term formula. Also calculate the \textbf{sum of the first 17 terms of this sequence}.

\texttt{Sequence: -3, 1, 5, 9, 13 ...}

\texttt{Formula: 4n-3 as the diffence between all the numbers is 4 and the sequence starts at -3}

\texttt{1n..17n = -3, 1, 5, 9, 13, 17, 21, 25, 29, 33, 37, 41, 45, 49, 53, 57, 61}

\newpage

\subsection{Sums to Terms and Infinity}
Find a formula for the \textsc{nth} term of this sequence and find the \textsc{10th} term using your \textsc{nth} term formula. Also calculate the \textbf{sum to the 5th term} and the \textbf{sum to infinity} of this sequence.

\subsubsection{nth Term}
\texttt{81, -27, 9, -3 ... = Xn+/-Y}
\texttt{Xn+/-Y}
\texttt{15n =}
\texttt{∞n =}

\subsubsection{10th Term}
\texttt{things}

\subsubsection{Sum to 5th Term}
\texttt{things}

\subsubsection{Sum to Infinity}
\texttt{things}

\newpage

\subsection{Equations}
Find the solution to
$$\sum_{r=1}^6 (3r - 2r^2 + r^3)$$

Substituting the Rs for 1s.
$$\sum_{r=1}^6 ((3 \times 1) - (2 \times 1^2) + (1^3))$$

Substituting the Rs for 2s.
$$\sum_{r=2}^6 ((3 \times 2) - (2 \times 2^2) + (2^3))$$

Substituting the Rs for 3s.
$$\sum_{r=3}^6 ((3 \times 3) - (2 \times 3^2) + (3^3))$$

Substituting the Rs for 4s.
$$\sum_{r=4}^6 ((3 \times 4) - (2 \times 4^2) + (4^3))$$

Substituting the Rs for 5s.
$$\sum_{r=5}^6 ((3 \times 5) - (2 \times 5^2) + (5^3))$$

Substituting the Rs for 6s.
$$\sum_{r=6}^6 ((3 \times 6) - (2 \times 6^2) + (6^3))$$

\newpage

Working out the brackets where R = 1.
$$\sum^6_{r=1} (3 - (2 \times 1) + 1)$$

Working out the brackets where R = 2.
$$\sum^6_{r=2} (6 - (2 \times 4) + 8)$$

Working out the brackets where R = 3.
$$\sum^6_{r=3} (9 - (2 \times 9) + 27)$$

Working out the brackets where R = 4.
$$\sum^6_{r=4} (12 - (2 \times 16) + 64)$$

Working out the brackets where R = 5.
$$\sum^6_{r=5} (15 - (2 \times 25) + 125)$$

Working out the brackets where R = 6.
$$\sum^6_{r=6} (18 - (2 \times 36) + 216)$$

\newpage

Final solution within the brackets where R = 1.
$$\sum^6_{r=1} (3 - 2 + 1)$$

Final solution within the brackets where R = 2.
$$\sum^6_{r=2} (6 - 8 + 8)$$

Final solution within the brackets where R = 3.
$$\sum^6_{r=3} (9 - 18 + 27)$$

Final solution within the brackets where R = 4.
$$\sum^6_{r=4} (12 - 32 + 64)$$

Final solution within the brackets where R = 5.
$$\sum^6_{r=5} (15 - 50 + 125)$$

Final solution within the brackets where R = 6.
$$\sum^6_{r=6} (18 - 72 + 216)$$

\newpage

Final solution without the brackets where R = 1.
$$\sum^6_{r=1} (2)$$

Final solution without the brackets where R = 2.
$$\sum^6_{r=2} (6)$$

Final solution without the brackets where R = 3.
$$\sum^6_{r=3} (18)$$

Final solution without the brackets where R = 4.
$$\sum^6_{r=4} (44)$$

Final solution without the brackets where R = 5.
$$\sum^6_{r=5} (90)$$

Final solution without the brackets where R = 6.
$$\sum^6_{r=6} (162)$$

\newpage

Therefore we need to add up all the numbers to get the final figure.
$$\sum (2 + 6 + 18 + 44 + 90 + 162)$$

This means that the final answer is:
$$322$$

\newpage

\subsection{Balls in Bags}

Five balls are in a bag, 3 are red and 2 are yellow. Once a ball is chosen at random the ball is put back into the bag and the bag is shaken well.

\begin{figure}[h!]
  \includegraphics[width=\linewidth]{probabilitytable.png}
  \caption{Probability Table.}
  \label{fig:chart1}
\end{figure}

\newpage

\subsubsection{What is the probability that a yellow ball is selected?}

The answer to the first question is $$\frac{2}{5}$$

This is because there is a total of five balls in the bag, and two of those are yellow meaning that there is a two in five chance of ever getting a yellow ball.

\subsubsection{What is the probability 2 yellow balls are selected consecutively?}

The answer to the second question is $$\frac{4}{10}$$

This is because there can be a maximum of 2 balls taken out if the answer is correct, and they both need to be yellow. Because of this, the chance of getting the first ball is two in five, as is the second.

\subsubsection{Draw a probability tree and use it to find the probability that a yellow ball is selected 4 times in a row.}

The answer to the third question is $$\frac{16}{625}$$

This is because 2 x 2 x 2 x 2 is 16 and 5 x 5 x 5 x 5 is 625. We multiply these numbers as the probabibility of getting one Yellow is two out of five, so therefore we need to multiply it by itself four times as we want to get the total value of all four balls being yellow.

\newpage

\subsection{Venn Diagrams}

\subsubsection{The Diagram}


\def\firstcircle{(0,0) circle (1.5cm)}
\def\secondcircle{(55:2cm) circle (1.5cm)}
\def\thirdcircle{(0:2cm) circle (1.5cm)}
\def\forthcircle{(5,0.5) circle (1.5cm)}

\begin{tikzpicture}
  \draw \firstcircle node[below] {$M$};
  \draw \secondcircle node [above] {$C$};
  \draw \thirdcircle node [below] {$E$};
  \draw \forthcircle node [below] {$N$};

  % Now we want to highlight the intersection of the first and the
  % second circle:

  \begin{scope}
    \fill[cyan] \firstcircle;
  \end{scope}

  \begin{scope}
    \fill[gray] \secondcircle;
  \end{scope}

  \begin{scope}
    \fill[teal] \thirdcircle;
  \end{scope}

  \begin{scope}
    \fill[olive] \forthcircle;
  \end{scope}

  \begin{scope}
    \clip \firstcircle;
    \fill[red] \secondcircle;
  \end{scope}

  \begin{scope}
    \clip \secondcircle;
    \fill[violet] \thirdcircle;
  \end{scope}

  \begin{scope}
    \clip \firstcircle;
    \fill[yellow] \thirdcircle;
  \end{scope}

  \begin{scope}
    \clip \firstcircle;
    \clip \secondcircle;
    \fill[green] \thirdcircle;
  \end{scope}
\end{tikzpicture}

In this Venn diagram, the:
\begin{enumerate}
  \item {\color{cyan}Cyan} part is for the students that only take Computer Science.
  \begin{itemize}
    \item This number is 70 students.
  \end{itemize}
  \item {\color{gray}Gray} part is for the students that only take Engineering.
  \begin{itemize}
    \item This number is 83 students.
  \end{itemize}
  \item {\color{teal}Teal} part is for the students that only take Mathematics.
  \begin{itemize}
    \item This number is 0 students.
  \end{itemize}
  \item {\color{olive}Olive} part is for the students that take none of the above.
  \begin{itemize}
    \item This number is 10 students.
  \end{itemize}
  \item {\color{yellow}Yellow} part is for the students that take both Computer Science and Mathematics.
  \begin{itemize}
    \item This number is 15 students.
  \end{itemize}
  \item {\color{violet}Violet} part is for the students that take both Engineering and Mathematics.
  \begin{itemize}
    \item This number is 12 students.
  \end{itemize}
  \item {\color{red}Red} part is for the students that take both Computer Science and Engineering.
  \begin{itemize}
    \item This number is 0 students.
  \end{itemize}
  \newpage
  \item {\color{green}Green} part is for the students that take all the subjects, Computer Science, Engineering and Mathematics.
  \begin{itemize}
    \item This number is 0 students.
  \end{itemize}
\end{enumerate}

\subsubsection{Probability: Computer Science but not Mathematics}

The total number of all of these students is

$$70 + 83 + 0 + 10 + 15 + 12 + 0 + 0 = 190$$

This means that the probability of a random Computer Science student that does not take Mathematics is

$$\frac{70}{190}$$

Which, similifed is:

$$\frac{35}{95}$$

And similfying it even more makes:

$$\frac{7}{19}$$

\subsubsection{Probability: Engineering with or without other subjects}

he total number of all of these students is:

$$70 + 83 + 0 + 10 + 15 + 12 + 0 + 0 = 190$$

This means that the probability of a random Engineering student that either does or does not take another subject is:

$$\frac{83 + 12}{190}$$

Which, similifed is:

$$\frac{95}{190}$$

And similfying it even more makes:

$$\frac{1}{2}$$

\newpage

\subsection{Betting Game}
A betting game involves one player throwing a six sided die to represent an attack and the other player throwing a four sided die to represent a defence.
\begin{enumerate}
  \item Draw a probability space diagram for this game.
  \item What is the most likely total score(s) from both dice?
  \item What is the least likely score(s) and why?
\end{enumerate}

\begin{tabular}{c}
    %do table.
\end{tabular}

\end{document}