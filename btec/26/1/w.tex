\documentclass[a4paper,10pt]{article}

\usepackage{graphicx}
\usepackage{tabu}
\usepackage{tikz}
\usepackage{amsmath}
\usepackage{amssymb}
\usepackage{pgfplots}
\usepackage{listings}
\usepackage{float}
\usepackage[a4paper]{geometry}

\begin{document}
\pgfplotsset{compat=1.14}
\title{Unit XXVI Assignment I}
\author{Nathan Windisch}
\date{Janurary 2017}
\maketitle
\pagenumbering{arabic}
\newpage
\tableofcontents
\newpage

  \section{Data}
    The following data is used in PII, PIII and PIV:
    \begin{equation*}
      \begin{split}
        &M =
        \begin{pmatrix}
          3 & -1\\
          4 &  2
        \end{pmatrix}
        \\
        &N =
        \begin{pmatrix}
           4 &  3\\
          -3 & -1
        \end{pmatrix}
        \\
        &P =
        \begin{pmatrix}
           1 & 3 & 5\\
          -1 & 2 & 4\\
          -3 & 4 & 3
        \end{pmatrix}
      \end{split}
      \qquad
      \qquad
      \qquad
      \qquad
      \qquad
      \qquad
      \qquad
      \qquad
      \begin{split}
        &Q =
        \begin{pmatrix}
          2 &  3 &  3\\
          4 &  4 & -2\\
          3 & -4 &  8
        \end{pmatrix}
        \\
        &R =
        \begin{pmatrix}
          9 &  2 & 6\\
          12 & -4 & 7
        \end{pmatrix}
        \\
        &S =
        \begin{pmatrix}
          -6 &  3\\
          -3 & -2\\
          -6 &  6
        \end{pmatrix}
      \end{split}
    \end{equation*}

  \newpage

  \section{Understanding Matrices and How Matrices Can Be Used To Represent Ordered Data}
    \subsection{Overview}
      A Matrix is a way of displaying data in an ordered format. Matrices are in a rectanguar format with cells comprised of rows and columns. Matrices can be used with one another to add, subtract and multiply. When writing out a matrix calculation, regular mathematical symbols are used, except for the full stop symbol (.), which is used for multiplcation of matrices. When multipling matrices, the order of which matrix comes first is key. \textsc{A . B} is not the same as \textsc{B . A}.
    \subsection{Order}
      The order of a matrix is very important. A matrix with the numbers

      \begin{align*}
        \begin{pmatrix}
          3 & 6\\
          9 & 5
        \end{pmatrix}
      \end{align*}

      will have a different outcome if manipulated with another number than a matrix with the numbers

      \begin{align*}
        \begin{pmatrix}
          6 & 3\\
          5 & 9
        \end{pmatrix}
      \end{align*}

      This means that if the order of any individual number is changed, the whole calculation could be invalidated.

    \subsection{Indecies}
    Indexes of matrices are selected subsections of a matrix. For instance, a 3x3 matrix may be like this;

    \begin{align*}
      \begin{pmatrix}
        9 & 2 & 8\\
        3 & 1 & 4\\
        7 & 6 & 5
      \end{pmatrix}
    \end{align*}

    But an index of the matrix would be only a small group, such as this 2x2 subsection.

    \begin{align*}
      \begin{pmatrix}
        3 & 1\\
        7 & 6
      \end{pmatrix}
    \end{align*}

    \subsection{Real World Applications}
      Matrices can be used in the real world in many different applications. One use of matrixes in the real world is the traits of a population of people, webpage rankings and cryptography. Without matrices, many real world applications would be hindered.

  \newpage

  \section{Adding and Subtracting Matrices}
    The following are the questions that need to be answered:
    \begin{enumerate}
      \item M + N
      \item P + Q
      \item M - N
      \item 3P
      \item 3P - 2Q
    \end{enumerate}
    The following are my answers to the question, with working out added to them as an intermediate step.

    \subsection{M + N}
      \begin{align*}
        M + N =
        \begin{pmatrix}
          3 & -1\\
          4 &  2
        \end{pmatrix}
        +
        \begin{pmatrix}
           4 &  3\\
          -1 & -1
        \end{pmatrix}
        =
        \begin{pmatrix}
          3 +  4 & -1 +  3\\
          4 + -3 &  2 + -1
        \end{pmatrix}
        =
        \begin{pmatrix}
          7 & 2\\
          1 & 1
        \end{pmatrix}
      \end{align*}
    \subsection{P + Q}
      \begin{align*}
        P + Q =
        \begin{pmatrix}
           1 & 3 & 5\\
          -1 & 2 & 4\\
          -1 & 4 & 3
        \end{pmatrix}
        +
        \begin{pmatrix}
          2 & 3 & 3\\
          4 & 4 & -2\\
          3 & -4 & 8
        \end{pmatrix}
        =
        \begin{pmatrix}
          1 + 2 & 3 + 3 & 5 + 3\\
          -1 + 4 & 2 + 4 & 4 + -2\\
          -3 + 3 & 4 + -4 & 3 + 8
        \end{pmatrix}
        =
        \begin{pmatrix}
          3 & 6 & 8\\
          3 & 6 & 2\\
          0 & 0 & 11
        \end{pmatrix}
      \end{align*}
    \subsection{M - N}
      \begin{align*}
        M - N =
        \begin{pmatrix}
          3 & -1\\
          4 &  2
        \end{pmatrix}
        -
        \begin{pmatrix}
           4 & 3\\
          -1 & -1
        \end{pmatrix}
        =
        \begin{pmatrix}
          3 - 4 & -1 - 3\\
          4 - -3 & 2 - -1
        \end{pmatrix}
        =
        \begin{pmatrix}
          3 - 4 & -1 - 3\\
          4 + 3 & 2 + 1
        \end{pmatrix}
        =
        \begin{pmatrix}
          -1 & -4\\
          7 & 3
        \end{pmatrix}
      \end{align*}

    \subsection{3P}
      \begin{align*}
        3P =
        3
        \begin{pmatrix}
          1 & 3 & 5\\
          -1 & 2 & 4\\
          -3 & 4 & 3\\
        \end{pmatrix}
        =
        \begin{pmatrix}
          3(1) & 3(3) & 3(5)\\
          3(-1) & 3(2) & 3(4)\\
          3(-3) & 3(4) & 3(3)
        \end{pmatrix}
        =
        \begin{pmatrix}
          3 & 9 & 15\\
          -3 & 6 & 12\\
          -9 & 12 & 9\\
        \end{pmatrix}
      \end{align*}

    \subsection{3P - 2Q}
    Due to the fact that I have already calculated 3P, I shall now only calculate 2Q and then add them together at the end.
      \begin{align*}
        2Q =
        2
        \begin{pmatrix}
          2 & 3 & 3\\
          4 & 4 & -2\\
          3 & -4 & 8
        \end{pmatrix}
        =
        \begin{pmatrix}
          2(2) & 2(3) & 2(3)\\
          2(4) & 2(4) & 2(-2)\\
          2(3) & 2(-4) & 2(8)
        \end{pmatrix}
        =
        \begin{pmatrix}
          4 & 6 + 6\\
          8 & 8 & -4\\
          6 & -8 & 16
        \end{pmatrix}
      \end{align*}

      Now I will perform 3P - 2Q now that I have calculated 2Q.
      \begin{align*}
        \begin{pmatrix}
          3 & 9 & 15\\
          -3 & 6 & 12\\
          -9 & 12 & 9
        \end{pmatrix}
        -
        \begin{pmatrix}
          4 & 6 & 6\\
          8 & 8 & -4\\
          6 & -8 & 16
        \end{pmatrix}
        =
        \begin{pmatrix}
          3 - 4 & 9 - 6 & 15 - 6\\
          -3 - 8 & 6 - 8 & 12 - -4\\
          -9 - 6 & 12 - -8 & 9 - 16
        \end{pmatrix}
        =
        \begin{pmatrix}
          -1 & 3 & 9\\
          -11 & -2 & 16\\
          -15 & 20 & -7
        \end{pmatrix}
      \end{align*}

  \newpage

  \section{Multiplying Matrices}
    The following are the questions that need to be answered:

    \begin{enumerate}
      \item M $\times$ N
      \item P $\times$ Q
      \item R $\times$ S
      \item S $\times$ R
    \end{enumerate}

    The following are my answers to the questions, along with the working out added to then as an intermediate step.
    \subsection{M $\times$ N}
      \begin{align*}
        \begin{pmatrix}
          3 & -1\\
          4 & 2
        \end{pmatrix}
        \begin{pmatrix}
          4 & 3\\
          -1 & -1
        \end{pmatrix}
        =
        \begin{pmatrix}
          (3 \times 4) + (-1 \times -3) & (3 \times 3) + (-1 \times -1)\\
          (4 \times 4) + (2 \times -3) & (4 \times 3) + (2 \times -1)
        \end{pmatrix}
      \end{align*}
      \begin{align*}
        =
        \begin{pmatrix}
          12 + 3 & 9 + 1\\
          16 + -6 & 12 + -2
        \end{pmatrix}
        =
        \begin{pmatrix}
          15 & 10\\
          10 & 10
        \end{pmatrix}
      \end{align*}

    \subsection{P $\times$ Q}
      \begin{align*}
        \begin{pmatrix}
          1 & 3 & 5\\
          -1 & 2 & 4\\
          -3 & 4 & 3
        \end{pmatrix}
        \begin{pmatrix}
          2 & 3 & 3\\
          4 & 4 & -2\\
          3 & -4 & 8
        \end{pmatrix}
        =
      \end{align*}
      \begin{footnotesize}
        \begin{align*}
          \begin{pmatrix}
            ( 1 \times 2) + (3 \times 4) + (5 \times 3) & ( 1 \times 3) + (3 \times 4) + (5 \times -4) & ( 1 \times 3) + (3 \times -2) + (5 \times 8)\\
            (-1 \times 2) + (2 \times 4) + (4 \times 3) & (-1 \times 3) + (2 \times 4) + (4 \times -4) & (-1 \times 3) + (2 \times -2) + (4 \times 8)\\
            (-3 \times 2) + (4 \times 4) + (3 \times 3) & (-3 \times 3) + (4 \times 4) + (3 \times -4) & (-3 \times 3) + (4 \times -2) + (3 \times 8)
          \end{pmatrix}
        \end{align*}
      \end{footnotesize}
      \begin{align*}
        =
        \begin{pmatrix}
          2 + 12 + 15 & 3 + 12 + -20 & 3 + -6 + 40\\
          -2 + 8 + 12 & -3 + 8 + -16 & -3 + -4 + 32\\
          -6 + 16 + 9 & -9 + 16 + -12 & -9 + -8 + 24
        \end{pmatrix}
        =
        \begin{pmatrix}
          29 & -5 & 37\\
          18 & -11 & 25\\
          19 & -6 & 7
        \end{pmatrix}
      \end{align*}

      \newpage

      \subsection{R $\times$ S}
        \begin{align*}
          \begin{pmatrix}
            9 & 2 & 6\\
            12 & -4 & 7
          \end{pmatrix}
          \begin{pmatrix}
            -6 & 3\\
            -3 & -2\\
            -6 & 6
          \end{pmatrix}
        \end{align*}
        \begin{align*}
          =
          \begin{pmatrix}
            ( 9 \times -6) + ( 2 \times -3) + (6 \times -6) & ( 9 \times 3) + ( 2 \times -2) + (6 \times 6)\\
            (12 \times -6) + (-4 \times -3) + (7 \times -6) & (12 \times 3) + (-4 \times -2) + (7 \times 6)
          \end{pmatrix}
        \end{align*}
        \begin{align*}
          =
          \begin{pmatrix}
             54 + -6 + -36 &  27 + -4 + 36\\
            -72 + 12 + -42 & -36 +  8 + 42
          \end{pmatrix}
          =
          \begin{pmatrix}
             -96 & 59\\
            -102 & -86
          \end{pmatrix}
        \end{align*}

      \subsection{S $\times$ R}
        \begin{align*}
          \begin{pmatrix}
            -6 & 3\\
            -3 & -2\\
            -6 & 6
          \end{pmatrix}
          \begin{pmatrix}
            9 & 2 & 6\\
            12 & -4 & 7
          \end{pmatrix}
        \end{align*}
        \begin{align*}
          =
          \begin{pmatrix}
            (-6 \times 9) + ( 3 \times 12) & (-6 \times 2) + ( 3 \times -4) & (-6 \times 6) + ( 3 \times 7)\\
            (-3 \times 9) + (-2 \times 12) & (-3 \times 2) + (-2 \times -4) & (-3 \times 6) + (-2 \times 7)\\
            (-6 \times 9) + ( 6 \times 12) & (-6 \times 2) + ( 6 \times -4) & (-6 \times 6) + ( 6 \times 7)
          \end{pmatrix}
        \end{align*}
        \begin{align*}
          =
          \begin{pmatrix}
            -54 +  48 & -12 + -12 & -36 +  21\\
            -27 + -24 & -6  +   8 & -18 + -14\\
            -54 +  72 & -12 + -24 & -36 +  42
          \end{pmatrix}
          =
          \begin{pmatrix}
            -18 & -24 & -15\\
            -51 &   2 & -32\\
             18 & -36 &   6
          \end{pmatrix}
        \end{align*}

    \newpage

    \section{Inverse and Transpose}
      \subsection{Inverse}
        When generating an Inverse Matrix, there are many steps to follow that can make the process seem difficult as they all need to be performed in the right in order to ensure that the calculations are done correctly. Fortunately, I shall show how to perform an Inverse function by breaking it down into simple steps. The first method I shall show is a general rule of thumb for generating matrices.
        \subsubsection{Generic Method}
          I shall start off this explaination by naming some terms and definining them. An inverse square is one where the original and an inverse of that are multiplied with one another to get the answer which will give you another matrix. This tertiary matrix is called an indentity matrix due to the fact that it contains nothing but zeros, apart from a diagonal line going from the top left corner down to the bottom right corner of the matrix. The following is an example of an identity matrix:

          \begin{align*}
            \begin{pmatrix}
              1 & 0 & 0 & \quad & 0\\
              0 & 1 & 0 & \quad & 0\\
              0 & 0 & 1 & \quad & 0\\
              \quad & \quad & \quad & \quad & 0\\
              0 & 0 & 0 & 0 & 1
            \end{pmatrix}
          \end{align*}
        \subsubsection{2x2 Method}
          Now that the generic method of formulating an Inverse Matrix, I shall now show the method for the 2x2 matrix:


          \paragraph{Determinate Generation}
            The first thing that is needed when trying to generate 2x2 Inverse Matrix is the determinate of the matrix. The determine is calculated by multiplying the bottom right and top left values together and then subtracting the top left and bottom right values from that other number. When displayed in a matrix, it can be seen as follows:

            (Please note that A will be surrounded by square brackets due to the fact that it is an absolute value.)

            \begin{align*}
              A =
              \begin{pmatrix}
                a & b\\
                c & d
              \end{pmatrix}
              = [A] = ad - cb
            \end{align*}

            To provide this method with some more grounded results, I shall substitute the algebraic letters with randomly generated numbers, as follows:

            \begin{align*}
              A =
              \begin{pmatrix}
                2 & 6\\
                7 & 9
              \end{pmatrix}
              = [A] = 2(9) - 7(6) = 18 - 42 = -24
                [A] = -24
            \end{align*}

          \paragraph{Inverse Formula Generation}
            After the determinate has been calculated, an inverse of the original formula can be found. To do this, all that is required is the switching of the bottom right and top left values of the matrix. After this, the other two values need to be swapped with the negatives of their original values and the final value is the product of the whole matrix by one over the determinate. To show this in a more visual format, I shall provide the general formula for this function:

            \begin{align*}
              A =
              \begin{pmatrix}
                a & b\\
                c & d
              \end{pmatrix}
              =
              A^{-1} = \frac{1}{[A]}
              \begin{pmatrix}
                d & -b\\
                -c & a
              \end{pmatrix}
              =
              \frac{1}{ad-bc}
              \begin{pmatrix}
                d & -b\\
                -c & a
              \end{pmatrix}
            \end{align*}
            Put into one equation, we can show this as:
            \begin{align*}
              \begin{pmatrix}
                \frac{1}{ad-bc}d & -\frac{1}{ad-bc}b\\
            		-\frac{1}{ad-bc}c & \frac{1}{ad-bc}a
              \end{pmatrix}
            \end{align*}

            To try to explain this even more, I shall substitute the algebraic valies for real numbers:

            \begin{align*}
              A =
              \begin{pmatrix}
                2 & 4\\
                1 & 9
              \end{pmatrix}
              =
              A^{-1} = \frac{1}{[A]}
              \begin{pmatrix}
                9 & -4\\
                -1 & 2
              \end{pmatrix}
              =
              \frac{1}{(2 \times 9) - (4 \times 1)}
              \begin{pmatrix}
                9 & -4\\
                -1 & 2
              \end{pmatrix}
            \end{align*}
            Put into one equation, we can show this as:
            \begin{align*}
              \begin{pmatrix}
                \frac{1}{(2 \times 9) - (4 \times 1)}9 & -\frac{1}{(2 \times 9) - (4 \times 1)}4\\
                -\frac{1}{(2 \times 9) - (4 \times 1)}1 & \frac{1}{(2 \times 9) - (4 \times 1)}2
              \end{pmatrix}
            \end{align*}

      \newpage

      \subsection{Transpose}
        Another way that matrices can be modified is by "Transposing" them. Transposing meerly swaps numbers within the matrix around, meaning that a 3x2 matrix would simply become a 2x3 matrix. I shall demonstrate how a transposition would work, using algebraic functions.

        \begin{align*}
          \begin{pmatrix}
            a & b & c\\
            d & e & f
          \end{pmatrix}
          ^T
           =
          \begin{pmatrix}
            a & d\\
            b & e\\
            c & f
          \end{pmatrix}
          ^T
        \end{align*}

        Now I shall use random numbers for a more practical demonstration:

        \begin{align*}
          \begin{pmatrix}
            1 & 5 & 3\\
            7 & 9 & 6
          \end{pmatrix}
          ^T
           =
          \begin{pmatrix}
            1 & 7\\
            5 & 9\\
            3 & 6
          \end{pmatrix}
          ^T
        \end{align*}

        \subsubsection{The Questions}
        The following are the questions that need to be answered:
        \begin{enumerate}
          \item M$^-1$
          \item N$^-1$
          \item P$^-1$
          \item Q$^-1$
        \end{enumerate}
        \begin{enumerate}
          \item M$^T$
          \item P$^T$
          \item R$^T$
        \end{enumerate}

        The following are my answers to the questions, along with the working out added to them as an intermediate step.

        For reference, when calculating the inverse of a matrix you need to calculate the inverse of each individual element.

        \subsection{M$^-1$}
          \begin{align*}
            M =
            \begin{pmatrix}
              3 & -1\\
              4 &  2
            \end{pmatrix}
            ^-1
            =
            \begin{pmatrix}
              \frac{1}{5} & \frac{1}{10}\\
             \frac{-2}{5} & \frac{3}{10}
            \end{pmatrix}
          \end{align*}

        \subsection{N$^-1$}
            \begin{align*}
              N =
              \begin{pmatrix}
                4 &  3\\
               -1 & -1
              \end{pmatrix}
              ^-1
              =
              \begin{pmatrix}
                1 &  3\\
               -1 & -4
              \end{pmatrix}
            \end{align*}


        \subsection{P$^-1$}
          \begin{align*}
            P =
            \begin{pmatrix}
              -1 & 24\\
              -3 & 43
            \end{pmatrix}
            ^-1
             =
            \begin{pmatrix}
              \frac{43}{29} & \frac{-24}{29}\\
              \frac{ 3}{29} & \frac{ -1}{29}
            \end{pmatrix}
          \end{align*}

        \subsection{Q$^-1$}
          \begin{align*}
            Q =
            \begin{pmatrix}
              -1 & 24\\
              -3 & 43
            \end{pmatrix}
            = Undefined
          \end{align*}
          \\
          This is due to the fact that the matrix needs to be a square for an inverse to be found.

        \subsection{M$^-T$}
          \begin{align*}
            M =
            \begin{pmatrix}
              3 & -1\\
              4 &  2
            \end{pmatrix}
            ^-T
             =
            \begin{pmatrix}
               3 & -1\\
              -1 &  2
            \end{pmatrix}
          \end{align*}

        \subsection{P$^-T$}
          \begin{align*}
            P =
            \begin{pmatrix}
              1 & 35\\
             -1 & 24\\
             -3 & 43
            \end{pmatrix}
            ^-T
             =
            \begin{pmatrix}
               1 & -1 & -3\\
              35 & 24 & 43
            \end{pmatrix}
          \end{align*}

        \subsection{R$^-T$}
          \begin{align*}
            R =
            \begin{pmatrix}
              2 &  3 & 3\\
              9 &  2 & 6\\
             12 & -4 & 7
            \end{pmatrix}
            ^-T
             =
            \begin{pmatrix}
               2 & 9 & 12\\
               3 & 2 & -4\\
               3 & 6 &  7
            \end{pmatrix}
          \end{align*}

    \newpage

    \section{Simulatious Equations}
      The following questions need to be answered:

      \begin{enumerate}
        \item 3x + 4y = 14
              2x - 7y = 11
        \item 6x + 2y = 24
              3x - 3y = 22
      \end{enumerate}

      The following are my answers to the questions, along with the working out added to them as an intermediate step.

      \subsection{3x + 4y = 14\\2x - 7y = 11}
      \[
        \begin{split}
          3x + 4y = 14\\2x - 7y = 11
        \end{split}
        \qquad
        =
        \qquad
        \begin{split}
          \begin{pmatrix}
            3 &  4\\2 & -7
          \end{pmatrix}
        \end{split}
        \begin{pmatrix}
          x\\y
        \end{pmatrix}
        =
        \begin{pmatrix}
          14\\11
        \end{pmatrix}
      \]
      This can then be formatted in the following way:
      \begin{equation*}
        \begin{pmatrix}
          x\\y
        \end{pmatrix}
        =
        \begin{pmatrix}
          3 & 4\\2 & -7
        \end{pmatrix}
        ^{-1}
        \begin{pmatrix}
          14\\11
        \end{pmatrix}
      \end{equation*}
	    The first thing that I will do is to work out the inverse of the following matrix:
      \begin{equation*}
	       \begin{pmatrix}
           3 &  4\\
           2 & -7
	       \end{pmatrix}
      \end{equation*}
      I shall do this as follows:
      \begin{equation*}
        \frac{1}{[A]}
        =
      \end{equation*}
      \begin{equation*}
        \begin{pmatrix}
          3 & 4\\2 & -7
        \end{pmatrix}
        =
      \end{equation*}
      \\
      \begin{equation*}
        \frac{1}{(3 \times -7) - (4 \times 2)}
        \begin{pmatrix}
          3 & 4\\2 & -7
        \end{pmatrix}
        =
      \end{equation*}
      \\
      \begin{equation*}
        \frac{1}{-29}
        \begin{pmatrix}
          -7 & 4\\2 & 3
        \end{pmatrix}
        =
        \frac{1}{29}
        \begin{pmatrix}
          -7 & -4\\-2 & 3
        \end{pmatrix}
        =
      \end{equation*}
      \\
      \begin{equation*}
        \begin{pmatrix}
          \frac{1}{-29} \times -7 & \frac{1}{-29} \times -4\\
          \frac{1}{-29} \times -2 & \frac{1}{-29} \times  3
        \end{pmatrix}
        =
      \end{equation*}
      \\
      \begin{equation*}
        A^-1 =
        \begin{pmatrix}
          \frac{7}{29} & \frac{4}{29}\\\frac{2}{29} & \frac{-3}{29}
        \end{pmatrix}
        \hspace{0.4cm}
      \end{equation*}
      \newpage

      Now, I shall multiply the previous matrix, which is an inverse, with the following matrix:
      \begin{equation*}
	       \begin{pmatrix}
           x\\y
	       \end{pmatrix}
         =
         \begin{pmatrix}
           \frac{7}{29} & \frac{4}{29}\\
           \frac{2}{29} & \frac{-3}{-29}
         \end{pmatrix}
         \begin{pmatrix}
           14\\11
         \end{pmatrix}

      \end{equation*}
      \begin{equation*}
        \begin{pmatrix}
          \frac{7}{29} & \frac{4}{29}\\
          \frac{2}{29} & \frac{-3}{-29}
        \end{pmatrix}
        \begin{pmatrix}
          
      \end{equation*}

\end{document}